\documentclass[a4paper,twoside,justified]{tufte-book}

\hypersetup{colorlinks}% uncomment this line if you prefer colored hyperlinks (e.g., for onscreen viewing)


%%
% Book metadata
\title[Workshop Artificial Life]{Workshop\\ Artificial Life}
\author[Wouter Bulten, Robert-Jan Drenth, Frank Dorssers]{Wouter Bulten, Robert-Jan Drenth \& Frank Dorssers}
\publisher{Symposium committee CognAC, ACAIS 2013}

%%
% If they're installed, use Bergamo and Chantilly from www.fontsite.com.
% They're clones of Bembo and Gill Sans, respectively.
%\IfFileExists{bergamo.sty}{\usepackage[osf]{bergamo}}{}% Bembo
%\IfFileExists{chantill.sty}{\usepackage{chantill}}{}% Gill Sans

%\usepackage{microtype}

%%
% Just some sample text
\usepackage{lipsum}

%%
% For nicely typeset tabular material
\usepackage{booktabs}

%%
% For graphics / images
\usepackage{graphicx}
\setkeys{Gin}{width=\linewidth,totalheight=\textheight,keepaspectratio}
\graphicspath{{graphics/}}

% The fancyvrb package lets us customize the formatting of verbatim
% environments.  We use a slightly smaller font.
\usepackage{fancyvrb}
\fvset{fontsize=\normalsize}

%%
% Prints argument within hanging parentheses (i.e., parentheses that take
% up no horizontal space).  Useful in tabular environments.
\newcommand{\hangp}[1]{\makebox[0pt][r]{(}#1\makebox[0pt][l]{)}}

%%
% Prints an asterisk that takes up no horizontal space.
% Useful in tabular environments.
\newcommand{\hangstar}{\makebox[0pt][l]{*}}

%%
% Prints a trailing space in a smart way.
\usepackage{xspace}

%%
% Some shortcuts for Tufte's book titles.  The lowercase commands will
% produce the initials of the book title in italics.  The all-caps commands
% will print out the full title of the book in italics.

% Prints the month name (e.g., January) and the year (e.g., 2008)
\newcommand{\monthyear}{%
  \ifcase\month\or January\or February\or March\or April\or May\or June\or
  July\or August\or September\or October\or November\or
  December\fi\space\number\year
}


% Prints an epigraph and speaker in sans serif, all-caps type.
\newcommand{\openepigraph}[2]{%
  %\sffamily\fontsize{14}{16}\selectfont
  \begin{fullwidth}
  \sffamily\large
  \begin{doublespace}
  \noindent\allcaps{#1}\\% epigraph
  \noindent\allcaps{#2}% author
  \end{doublespace}
  \end{fullwidth}
}

% Inserts a blank page
\newcommand{\blankpage}{\newpage\hbox{}\thispagestyle{empty}\newpage}

\usepackage{units}

% Typesets the font size, leading, and measure in the form of 10/12x26 pc.
\newcommand{\measure}[3]{#1/#2$\times$\unit[#3]{pc}}

\usepackage{color}
\usepackage{listings}
\lstset{
	tabsize=4,
	language=matlab,
       % basicstyle=\scriptsize,
        %upquote=true,
        aboveskip={1.5\baselineskip},
        columns=fixed,
        showstringspaces=false,
        extendedchars=true,
        breaklines=true,
        prebreak = \raisebox{0ex}[0ex][0ex]{\ensuremath{\hookleftarrow}},
	frame=none,
        showtabs=false,
        showspaces=false,
        showstringspaces=false,
        identifierstyle=\ttfamily,
        keywordstyle=\color[rgb]{0,0,1},
        commentstyle=\color[rgb]{0.133,0.545,0.133},
        stringstyle=\color[rgb]{0.627,0.126,0.941},
	language=Python
}
\usepackage[parfill]{parskip}

\newcommand{\instruct}[1]{
\begin{itemize}[>]
	\item \textbf{#1}
\end{itemize}
}

\begin{document}


% r.3 full title page
\maketitle


% v.4 copyright page
\newpage
\begin{fullwidth}
~\vfill
\thispagestyle{empty}
\setlength{\parindent}{0pt}
\setlength{\parskip}{\baselineskip}
Copyright \copyright\ \the\year\ \thanklessauthor

\par\smallcaps{Published by \thanklesspublisher}


\par Documentation Workshop Artificial Life, ACAIS 2013 by Wouter Bulten, Robert-Jan Drenth \& Frank Dorssers is licensed under a Creative Commons Attribution-NonCommercial-ShareAlike 3.0 Unported License. To view a copy of this license, visit \url{http://creativecommons.org/licenses/by-nc-sa/3.0/}.\index{license}

\par\textit{First printing, \monthyear}
\end{fullwidth}

% r.5 contents
\tableofcontents
\setlength\parindent{0pt}
\chapter*{Introduction}

During ACAIS 2013 a workshop will be hosted which can be joined by all attendees who can bring a laptop (working together is of course allowed). The workshop will focus on becoming familiar with the 3D multi-agent simulation environment Breve. You will learn the basics of creating agents and letting them move in an artificial environment. After this workshop you will be able to create your own simulations using Breve.

This manual will guide you in creating your first simulation. You will create a environment were agents have to collect food. Your simulation is finished when all food sources are stacked together. There is one limitation, agents may not communicate with each other. If you have time left you can extend the simulation by introducing two competing groups. Finally it will look something like this:

\includegraphics{endproduct}

We hope you will like this workshop and get a good grasp of creating multi-agent and artificial life simulations.\\[1cm]

\textit{Wouter Bulten, Frank Dorssers, Robert-Jan Drenth}
%%
% Start the main matter (normal chapters)
\mainmatter
\chapter{Setting up Breve}
The program essential to this workshop is Breve. In this guide we will be guiding you through all necessary steps. 

\section{Windows}
In this section it will be explained how to get Breve to work under Windows.
\subsection{Installation}
The files will have to be extracted to a location of your choosing. 

\subsection{Setting paths}
You will have to set two paths. One for Breve and one for the Python version that comes with Breve.

There are two possibilities here. The first one is temporary and has to be run every time you start a new command prompt. It consists of the following two lines:
\begin{lstlisting}
set BREVE_CLASS_PATH=<breve_path>\lib\classes
set PYTHONPATH=\%PYTHONPATH\%;<breve_path>\lib\python2.3
\end{lstlisting}
If, for example, you have copied Breve to `C:\\breve\_2.7.2' it would look like this:
\begin{lstlisting}
set BREVE_CLASS_PATH=C:\breve_2.7.2\lib\classes
set PYTHONPATH=\%PYTHONPATH\%;C:\breve_2.7.2\lib\python2.3
\end{lstlisting}

There is an alternative version which is more permanent, but can easily be undone once it is not required anymore. This is done by changing the environment veriables in the Advanced System Settings:
\begin{lstlisting}
Ctrl+R --> Fill in "control sysdm.cpl" --> Press enter to run it --> Advanced --> Environment Variables
\end{lstlisting}
% Beter alternatief vinden dan lstlisting

Here you will see two areas, one are the user variables and the other are the system variables. We will be adding two variables to the last one. Simply press "New..." and enter the following information, substituting <breve\_path> with the actual path on your pc:
\begin{lstlisting}
Variable name: BREVE_CLASS_PATH
Variable value: <breve_path>\lib\classes
\end{lstlisting}
\begin{lstlisting}
Variable name: PYTHONPATH
Variable value: <breve_path>\lib\python2.3
\end{lstlisting}
% Nog toevoegen hoe het zit als je al Python in gebruik hebt en al paden hebt ignesteld

\section{Mac OS X}

\chapter{Step 1:  Creating a simulation}
\begin{lstlisting}[language=Python]
Put your code here.
\end{lstlisting}

\chapter{Step 2: Basic environment}
Our environment is now totally empty. To resolve this we will add a simple floor.

\instruct{In the SimulationController, add the following code in the \textit{\_\_init\_\_} function\footnote{Make sure that you place it after the call to the parent constructor.}:}

\begin{lstlisting}[language=Python]

# Create a floor for the world
self.floor = breve.Floor()
self.floor.setTextureImage(None)
            
\end{lstlisting}

These two lines add a new floor object (which has a ground plane around $Y=0$). We set the texture to None to prevent the use of the (ugly) default texture.

To improve the visual aspects of the simulation we enable lighting and shadows. This should only be used for demonstration purposes as it makes the simulation run slower.

\instruct{Add the following snippet after the code for the floor:}

\begin{lstlisting}[language=Python]
# Set display settings
self.enableLighting()
self.enableShadows()
self.moveLight(breve.vector(80,100,0))
self.enableReflections()
self.enableSmoothDrawing()
\end{lstlisting}

\chapter{Step 3: Creating an agent}
After a floor has been added we can add more objects to our world. We will start with agents.

\instruct{Create a new file agents.py}

As we are going to use functions (and classes) of the breve engine, we must import it at the top of our file.

\begin{lstlisting}[language=Python]
import breve
\end{lstlisting}

\instruct{Add the import statement to the top of our agents file.}

\section{Creating a simple agent}

We will start with a very simple agent which will do nothing but stand still. Every agent must extend the \textit{breve.Mobile} class in order for it to move and interact.

Just as with our controller, we have to add a constructor and an \textit{iterate} function.

\begin{lstlisting}[language=Python]
class SimpleAgent (breve.Mobile):

	def __init__(self):
		breve.Mobile.__init__(self)

		print "Created agent"

	def iterate(self):
		None
\end{lstlisting}

\instruct{Add the class definition above to the agents file.}

\instruct{Try to run the simulation and see what happens.}

As you can see nothing happened yet. This is because we still have to add our agent to the environment. This is done through the simulation controller.

\begin{lstlisting}[language=Python]
self.agent = agents.SimpleAgent()
\end{lstlisting}

\instruct{Add the following snippet to the \textit{\_\_init\_\_} function of the controller, just below the lighting settings.}

We now have a single agent in our environment.

\chapter{Step 4: Adding food}
The second object we will be adding is food. This gives our agents something that they can interact with.

\instruct{Create a new file food.py}

As we are going to use functions (and classes) of the breve engine, we must import it at the top of our file.

\begin{lstlisting}[language=Python]
import breve
\end{lstlisting}

\instruct{Add the import statement to the top of our agents file.}

\section{Creating the food object}

Our initial version of the food is very simple, it will simply stay on the field. Every food source must extend the \textit{breve.Mobile} class. Food does not have to move, but we want to be able to interact with it.

Just as with our controller and agent, we have to add a constructor and an \textit{iterate} function.

\begin{lstlisting}[language=Python]
class SimpleFood (breve.Mobile):

    def __init__(self):
        breve.Mobile.__init__(self)

    def iterate(self):
        None
\end{lstlisting}

\instruct{Add the class definition above to the food file.}

From the previous section we know that this is not enough. We will also have to add the food to the controller

\begin{lstlisting}[language=Python]
self.SimpleFood = food.SimpleFood()
\end{lstlisting}

\instruct{Add the snippet to the \textit{\_\_init\_\_} function of the controller, just below the lighting settings.}

The food should now have been added to the simulation.

\instruct{Try to run the simulation and see what happens.}
As you can see you have now added a single food object to the field, though it does nothing yet.

\section{Location}
Our food is always at the same location when running the program. We want it to be more random. This can be done with the following function.
\begin{fullwidth}
\begin{lstlisting}[language=Python]
def randomizedLocation(self):
    randomLoc = breve.randomExpression(2 * breve.vector(20,20,20)) - breve.vector(20,20,20)
    self.move(randomLoc)
\end{lstlisting}
\end{fullwidth}
\instruct{Add the function above to the food class}

Now we just have to call this function when we create the food.
\begin{lstlisting}[language=Python]
self.randomizedLocation()
\end{lstlisting}

\instruct{Add the line above to the \textit{\_\_init\_\_} function of the food}

\section{Appearance}
We will again define one shape for all food sources.

\instruct{Add the following to the constructor of the SimulationController}
\begin{fullwidth}
\begin{lstlisting}[language=Python]
# Create a shape for the food sources
self.foodShape = breve.createInstances(breve.Sphere, 1).initWith(0.5)
\end{lstlisting}
\end{fullwidth}
This will create a small sphere. To be able to use this shape in our agent we add a getter method to the controller.

\instruct{Add the following getter to the simulation controller:}
\begin{lstlisting}[language=Python]
def getFoodShape(self):
    return self.foodShape
\end{lstlisting}

Now it is very easy to change the shape of the food. We change it with the following line of code:
\begin{lstlisting}[language=Python]
# Set the shape of the food source
self.setShape(self.controller.getFoodShape())
\end{lstlisting}
\instruct{Try to change the shape of the food by adding the code to the constructor of the food. Test it by running the simulation, has the shape changed?}

\section{Creating more food}
A single food object is not much for all our agents, so we want to add some more. We first added a single food object to the SimulationController with the following code:
\begin{lstlisting}[language=Python]
self.SimpleFood = food.SimpleFood()
\end{lstlisting}
We now want more food objects, this can be done with the following code:
\begin{lstlisting}[language=Python]
breve.createInstances(food.SimpleFood, 3)
\end{lstlisting}
\instruct{Replace the first piece of code with the second piece of code to create more food objects}

\chapter{Step 5: An extra dimension}
As an intermediate step, we are going to switch from a 2D plane to a 3D space to illustrate how easy it is to do so in Breve. Additionally, it will gives us a much nicer view of the final result when we get there. 

The only changes we need to make are to the \textit{agent} and \textit{food} classes; the \textit{wanderer.WanderingAgent} class which our RandomAgent extends is able to handle 2D and 3D spaces just as easily without requiring any changes in the code. 

\instruct{Change the \textit{y-value} of wander range of the RandomAgent from 0 to 20}

\begin{fullwidth}
\begin{lstlisting}[language=Python]
self.setWanderRange(breve.vector(20, 20, 20))
\end{lstlisting}
\end{fullwidth}

\instruct{Change the food.SimpleFood.randomizedLocation() function so that also a random value is taken for the y-axis}

\begin{fullwidth}
\begin{lstlisting}[language=Python]
randomLoc = breve.randomExpression(2 * breve.vector(20,20,20)) - breve.vector(20,20,20)
\end{lstlisting}
\end{fullwidth}

If you run the simulation now, you will notice how the food sources are scattered in 3D space and how the agents will move through them.

\chapter{Application: Gatherers}
Now that we have created the basics of our \textit{Artificial Life} simulation, \textit{agents} and \textit{food sources}, it is time to make these interact since currently nothing is happening and agents just wander around aimlessly. To make things more interesting we will  implement a working \textit{Gatherers} simulation. 

In the Gatherers-simulation agents will move around freely in a world that is rich in food. Every time an agents encounters some food, it will pick it up and carry it with him until it comes across another food source. When it does that, it will drop the first food source next to the second and move on. Eventually, an agent will gather all food and put it all together in one big pile. 

Naturally, it can take a while before an agent has done this, especially if it walks around randomly like our agents. A group of agents will be able to achieve this task faster if they work together, which is why you created 10 of them.



\section{Adding interaction between food and agents}
Having food that does nothing is not interesting, we want to be able to interact with it. When an agent touches a food source he should pick it up and carry it for a while, before dropping it again.

In our case a food source can only be carried by one agent. So let us start with that. We will have to add an owner to the food to see if it is being carried.

\begin{lstlisting}
self.owner = None
\end{lstlisting}

\instruct{Add the line initializing the owner to the \textit{\_\_init\_\_} function of the food}
Now we want to be able to get and set this value from other classes. This can be done by adding a getter and a setter.
\begin{lstlisting}
def setOwner(self, o):
    self.owner = o

def getOwner(self):
    return self.owner
\end{lstlisting}
\instruct{Add the two functions to the food file}
That was all we had to change to the food file. From now on we will be working in the agent file. First we will add whether an agent is carrying an object or not.
\begin{lstlisting}
# Store a possible food source
self.carrying = None
\end{lstlisting}

\instruct{Add the line initializing the carrying status to the \textit{\_\_init\_\_} function of the agent}
First we will define what happens if an agent collides with a food source. There are several possibilities. The food you collide with could already be carried by someone or you have recently collided with another food source, in these two cases you do nothing. If you are carrying food and you collide with a food source that no one is carrying, you place your current food on the ground. If you are not carrying food and none of the above possibilities apply, you will pick the food up.
\begin{lstlisting}
def collisionWithFood(self, f):
    # Checks if the food is already being carried
    if(f.getOwner()):
        return

    # We want two iterations without an collision
    if(self.collidedTimer > 0):
        self.collidedTimer = 2
        return

    # Set the timer for the collision
    self.collidedTimer = 2

    if(self.carrying != None):
        self.placeFoodObject(self.carrying, f)
        self.carrying = None
        return

    self.carrying = f
    self.carrying.setOwner(self)
\end{lstlisting}

\instruct{Add the collision function to the RandomAgent class}
You might have noticed that we talked about placing food, something we will implement now. The food you are carrying is placed near the food you collided with.

\begin{fullwidth}
\begin{lstlisting}
def placeFoodObject(self, ownFood, placedFood):

    location = placedFood.getLocation()
    location = location + ( breve.randomExpression( breve.vector( 2, 2, 2 ) ) - breve.vector( 1, 1, 1 ) )

    ownFood.move(location)
    ownFood.setOwner(None)
\end{lstlisting}
\end{fullwidth}

\instruct{Add the placeFoodObject function above to the RandomAgent class}
In the collision function we used a timer which represents how long you have not touched any food. This is so you do not pick up the same object right away. However we have not initialized this variable yet.
\begin{fullwidth}
\begin{lstlisting}
# Timer to prevent / to inhibit the collision behavior
self.collidedTimer = 0
\end{lstlisting}
\end{fullwidth}

\instruct{Add the line initializing the timer to the \textit{\_\_init\_\_} function of the agent}
Changing the timer will happen in the iterator, so that every refresh this value is decreased.

\begin{lstlisting}
self.collidedTimer -= 1
\end{lstlisting}

\instruct{Add the line that decreases the timer to the \textit{iterate} function, before the iterate call}
At the moment we have all necessary functions to describe what happens when you collide with food. We are still missing what actually happens to the food while you are carrying it. When you are carrying food the location of the food should be adjusted along with your location.
\begin{fullwidth}
\begin{lstlisting}
if(self.carrying):
    self.carrying.move(self.getLocation() - breve.vector(1,0,0))
\end{lstlisting}
\end{fullwidth}

\instruct{Add the if and its body to the \textit{iterate} function of the agent}
We now have all necessary functions to interact with and carry food. But our agents do not know yet when they actually collide with food. For this we will use a standard breve function. First you give the object that it is colliding with and next you give the function that should be executed in case of collision.
\begin{fullwidth}
\begin{lstlisting}
# Setup the collision
self.handleCollisions('SimpleFood', 'collisionWithFood')
\end{lstlisting}
\end{fullwidth}
\instruct{Add the breve function and its parameters to the \textit{\_\_init\_\_} function of the agent}

Make sure that you create the food sources before you create the agents. Otherwise the agents will not know what the \textit{SimpleFood} object is.



\section{Creating two groups: Red versus Blue}
Now that we have finished creating a basic Gatherers simulation, it is time to go one step further. When creating an Artificial Life simulation, most of the time you will want to create different conditions of the environment and compare them to each other. We won't really make different conditions, but we will create two groups of agents who will compete with each other in the same world: the \textit{Red Agents} and the \textit{Blue Agents}.

First, we will create two new classes, \textit{RedAgent}  and \textit{BlueAgent} underneath our RandomAgent. These new types of agent will extend our RandomAgent. Then we need a property \textit{group} to indicate to which group they belong. We will also give them a colour so we can distinguish them from each other ourself when we watch the simulation. The color can be specified with the setColor(r, g, b) function where r, g, b are values between 0 and 1 for \textit{red, green} and \textit{blue} respectively. 

These extra properties should be set in the constructor. Naturally, these classes will also need an iterate function. However, these will do nothing special so simply calling the iterate function of the superclass is sufficient. 

\instruct{Add the RedAgent and BlueAgent classes below the RandomAgent class}

\begin{lstlisting}[language=Python]
class BlueAgent (RandomAgent):
	def __init__(self):
		RandomAgent.__init__(self)
		self.setColor(breve.vector(0.2,0.2,0.8))
		self.group = 1

	def iterate(self):
		RandomAgent.iterate(self)


class RedAgent (RandomAgent):
	def __init__(self):
		RandomAgent.__init__(self)
		self.setColor(breve.vector(0.8,0.2,0.2))
		self.group = 2

	def iterate(self):
		RandomAgent.iterate(self)
\end{lstlisting}

We will also want to give the RandomAgent a default group it belongs, just to make sure that we will not brake the interaction between food and agents in a bit.

\instruct{In the constructor of RandomAgent, add a line stating that it belongs to group 0}

\begin{lstlisting}[language=Python]
self.group = 0
\end{lstlisting}
\end{document}

