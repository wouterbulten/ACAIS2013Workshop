\documentclass{tufte-book}

\hypersetup{colorlinks}% uncomment this line if you prefer colored hyperlinks (e.g., for onscreen viewing)

%%
% Book metadata
\title[Workshop Artificial Life]{Workshop\\ Artificial Life}
\author[Wouter Bulten, Robert-Jan Drenth, Frank Dorssers]{Wouter Bulten, Robert-Jan Drenth, Frank Dorssers}
\publisher{Symposium committee CognAC, ACAIS 2013}

%%
% If they're installed, use Bergamo and Chantilly from www.fontsite.com.
% They're clones of Bembo and Gill Sans, respectively.
%\IfFileExists{bergamo.sty}{\usepackage[osf]{bergamo}}{}% Bembo
%\IfFileExists{chantill.sty}{\usepackage{chantill}}{}% Gill Sans

%\usepackage{microtype}

%%
% Just some sample text
\usepackage{lipsum}

%%
% For nicely typeset tabular material
\usepackage{booktabs}

%%
% For graphics / images
\usepackage{graphicx}
\setkeys{Gin}{width=\linewidth,totalheight=\textheight,keepaspectratio}
\graphicspath{{graphics/}}

% The fancyvrb package lets us customize the formatting of verbatim
% environments.  We use a slightly smaller font.
\usepackage{fancyvrb}
\fvset{fontsize=\normalsize}

%%
% Prints argument within hanging parentheses (i.e., parentheses that take
% up no horizontal space).  Useful in tabular environments.
\newcommand{\hangp}[1]{\makebox[0pt][r]{(}#1\makebox[0pt][l]{)}}

%%
% Prints an asterisk that takes up no horizontal space.
% Useful in tabular environments.
\newcommand{\hangstar}{\makebox[0pt][l]{*}}

%%
% Prints a trailing space in a smart way.
\usepackage{xspace}

%%
% Some shortcuts for Tufte's book titles.  The lowercase commands will
% produce the initials of the book title in italics.  The all-caps commands
% will print out the full title of the book in italics.

% Prints the month name (e.g., January) and the year (e.g., 2008)
\newcommand{\monthyear}{%
  \ifcase\month\or January\or February\or March\or April\or May\or June\or
  July\or August\or September\or October\or November\or
  December\fi\space\number\year
}


% Prints an epigraph and speaker in sans serif, all-caps type.
\newcommand{\openepigraph}[2]{%
  %\sffamily\fontsize{14}{16}\selectfont
  \begin{fullwidth}
  \sffamily\large
  \begin{doublespace}
  \noindent\allcaps{#1}\\% epigraph
  \noindent\allcaps{#2}% author
  \end{doublespace}
  \end{fullwidth}
}

% Inserts a blank page
\newcommand{\blankpage}{\newpage\hbox{}\thispagestyle{empty}\newpage}

\usepackage{units}

% Typesets the font size, leading, and measure in the form of 10/12x26 pc.
\newcommand{\measure}[3]{#1/#2$\times$\unit[#3]{pc}}

\usepackage{color}
\usepackage{listings}
\lstset{
	tabsize=4,
	language=matlab,
       % basicstyle=\scriptsize,
        %upquote=true,
        aboveskip={1.5\baselineskip},
        columns=fixed,
        showstringspaces=false,
        extendedchars=true,
        breaklines=true,
        prebreak = \raisebox{0ex}[0ex][0ex]{\ensuremath{\hookleftarrow}},
	frame=none,
        showtabs=false,
        showspaces=false,
        showstringspaces=false,
        identifierstyle=\ttfamily,
        keywordstyle=\color[rgb]{0,0,1},
        commentstyle=\color[rgb]{0.133,0.545,0.133},
        stringstyle=\color[rgb]{0.627,0.126,0.941},
	language=Python
}

\begin{document}


% r.3 full title page
\maketitle


% v.4 copyright page
\newpage
\begin{fullwidth}
~\vfill
\thispagestyle{empty}
\setlength{\parindent}{0pt}
\setlength{\parskip}{\baselineskip}
Copyright \copyright\ \the\year\ \thanklessauthor

\par\smallcaps{Published by \thanklesspublisher}


\par Licensed under the Apache License, Version 2.0 (the ``License''); you may not
use this file except in compliance with the License. You may obtain a copy
of the License at \url{http://www.apache.org/licenses/LICENSE-2.0}. Unless
required by applicable law or agreed to in writing, software distributed
under the License is distributed on an \smallcaps{``AS IS'' BASIS, WITHOUT
WARRANTIES OR CONDITIONS OF ANY KIND}, either express or implied. See the
License for the specific language governing permissions and limitations
under the License.\index{license}

\par\textit{First printing, \monthyear}
\end{fullwidth}

% r.5 contents
\tableofcontents

\chapter*{Introduction}
The program essential to this workshop is Breve. In this guide we will be guiding you through all necessary steps. 

\section{Windows}
In this section it will be explained how to get Breve to work under Windows.
\subsection{Installation}
The files will have to be extracted to a location of your choosing. 

\subsection{Setting paths}
You will have to set two paths. One for Breve and one for the Python version that comes with Breve.

There are two possibilities here. The first one is temporary and has to be run every time you start a new command prompt. It consists of the following two lines:
\begin{lstlisting}
set BREVE_CLASS_PATH=<breve_path>\lib\classes
set PYTHONPATH=\%PYTHONPATH\%;<breve_path>\lib\python2.3
\end{lstlisting}
If, for example, you have copied Breve to `C:\\breve\_2.7.2' it would look like this:
\begin{lstlisting}
set BREVE_CLASS_PATH=C:\breve_2.7.2\lib\classes
set PYTHONPATH=\%PYTHONPATH\%;C:\breve_2.7.2\lib\python2.3
\end{lstlisting}

There is an alternative version which is more permanent, but can easily be undone once it is not required anymore. This is done by changing the environment veriables in the Advanced System Settings:
\begin{lstlisting}
Ctrl+R --> Fill in "control sysdm.cpl" --> Press enter to run it --> Advanced --> Environment Variables
\end{lstlisting}
% Beter alternatief vinden dan lstlisting

Here you will see two areas, one are the user variables and the other are the system variables. We will be adding two variables to the last one. Simply press "New..." and enter the following information, substituting <breve\_path> with the actual path on your pc:
\begin{lstlisting}
Variable name: BREVE_CLASS_PATH
Variable value: <breve_path>\lib\classes
\end{lstlisting}
\begin{lstlisting}
Variable name: PYTHONPATH
Variable value: <breve_path>\lib\python2.3
\end{lstlisting}
% Nog toevoegen hoe het zit als je al Python in gebruik hebt en al paden hebt ignesteld

\section{Mac OS X}

%%
% Start the main matter (normal chapters)
\mainmatter

\chapter{Step 1:  Creating a simulation}
\begin{lstlisting}[language=Python]
Put your code here.
\end{lstlisting}

\chapter{Step 2: Basic environment}
Our environment is now totally empty. To resolve this we will add a simple floor.

\instruct{In the SimulationController, add the following code in the \textit{\_\_init\_\_} function\footnote{Make sure that you place it after the call to the parent constructor.}:}

\begin{lstlisting}[language=Python]

# Create a floor for the world
self.floor = breve.Floor()
self.floor.setTextureImage(None)
            
\end{lstlisting}

These two lines add a new floor object (which has a ground plane around $Y=0$). We set the texture to None to prevent the use of the (ugly) default texture.

To improve the visual aspects of the simulation we enable lighting and shadows. This should only be used for demonstration purposes as it makes the simulation run slower.

\instruct{Add the following snippet after the code for the floor:}

\begin{lstlisting}[language=Python]
# Set display settings
self.enableLighting()
self.enableShadows()
self.moveLight(breve.vector(80,100,0))
self.enableReflections()
self.enableSmoothDrawing()
\end{lstlisting}

\end{document}

