\section{Creating two groups: Red versus Blue}
Now that we have finished creating a basic Gatherers simulation, it is time to go one step further. When creating an Artificial Life simulation, most of the time you will want to create different conditions of the environment and compare them to each other. We won't really make different conditions, but we will create two groups of agents who will compete with each other in the same world: the \textit{Red Agents} and the \textit{Blue Agents}.

First, we will create two new classes, \textit{RedAgent}  and \textit{BlueAgent} underneath our RandomAgent. These new types of agent will extend our RandomAgent. Then we need a property \textit{group} to indicate to which group they belong. We will also give them a colour so we can distinguish them from each other ourself when we watch the simulation. The color can be specified with the setColor(r, g, b) function where r, g, b are values between 0 and 1 for \textit{red, green} and \textit{blue} respectively. 

These extra properties should be set in the constructor. Naturally, these classes will also need an iterate function. However, these will do nothing special so simply calling the iterate function of the superclass is sufficient. 

\instruct{Add the RedAgent and BlueAgent classes below the RandomAgent class}

\begin{lstlisting}[language=Python]
class BlueAgent (RandomAgent):
	def __init__(self):
		RandomAgent.__init__(self)
		self.setColor(breve.vector(0.2,0.2,0.8))
		self.group = 1

	def iterate(self):
		RandomAgent.iterate(self)


class RedAgent (RandomAgent):
	def __init__(self):
		RandomAgent.__init__(self)
		self.setColor(breve.vector(0.8,0.2,0.2))
		self.group = 2

	def iterate(self):
		RandomAgent.iterate(self)
\end{lstlisting}

We will also want to give the RandomAgent a default group it belongs, just to make sure that we will not brake the interaction between food and agents in a bit.

\instruct{In the constructor of RandomAgent, add a line stating that it belongs to group 0}

\begin{lstlisting}[language=Python]
self.group = 0
\end{lstlisting}

Now that we created two new classes for the two groups of agents, we will have to make sure they are actually going to compete with each other (by stealing each other's food). We can do this by taking into account the group an agent belongs to when it interacts with a food source. 

The first step is to mark the food that an agent picks up as belonging to the group of that agent. In order to do this we must first update the \textit{SimpleFood} class so that it supports belonging to a group. We do this by adding a property \textit{group} in the constructor, just like we did for the agents. As all food starts off as neutral, it will belong to group 0. 

\instruct{In the constructor of SimpleFood, add a line stating that it belongs to group 0 (it is still neutral)}

\begin{lstlisting}[language=Python]
self.group = 0
\end{lstlisting}

Of course, we will also want to be able to check to what group food belongs to and change it. Thus, we must create a getter and setter function for the group-property.

\instruct{Add a getter and setter for the group-property at the bottom of the class}

\begin{lstlisting}[language=Python]
def setGroup(self, o):
	self.group = o

def getGroup(self):
	return self.group
\end{lstlisting}

Now we should make use of this property 